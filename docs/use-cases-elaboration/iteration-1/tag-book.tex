\documentclass{article}

\title{Use Case Elaboration: Tag Book}
\author{Nicholas Riesen}

\pagenumbering{gobble}

\begin{document}
\maketitle

\section*{Use Case}
 \begin{tabular}{l l}
 
Use Case:     & Tag Book \\
Actor:        & User     \\
Precondition: & None     \\
\end{tabular}

\section*{Main Success Scenario:}
\begin{enumerate}
    \item User provides a book title and a tag
    \item The system links the book to the tag and acknowledges successful completion of the operation
\end{enumerate}

\section*{Extensions}
\begin{itemize}
    \item[1a.] Tag does not exist \\
    - System creates tag and links the book and new tag

    \item[1b.] Book does not exist \\
    - System returns operation unsuccessful
    
\end{itemize}

\section*{Information Expert}
The information expert for this use case is the user. The user has direct access to all of the tags he/she has created and the books. These relationships make the user the most logical point to implement this functionality.

The method tagBook() will create a new instance of tag if needed, and link the requested tag and book, and put all changes into the database.

\end{document}
