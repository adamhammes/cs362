\documentclass{article}

\title{Use Case Elaboration - Add Version}
\author{Adam Hammes}

\pagenumbering{gobble}

\begin{document}
\maketitle

\section*{Use Case}

\begin{tabular}{l l}
  Use Case:     & Add Version \\
  Actor:        & User        \\
  Precondition: & None        \\
\end{tabular}

\section*{Main Success Scenario:}

\begin{enumerate}
  \item User provides user id, the location of the file, type of version
    to be added, and the unique id for the book.
  \item System acknowledges that the version has been added.
\end{enumerate}

\section*{Extensions}

\begin{enumerate}
  \item [1a.] File is not at given location

    - System indicates that the operation failed.

  \item [1b.] Version of file already exists

    - System confirms overwriting of old version.

  \item [1c.] Book id does not exist.

    - System indicates that the operation failed.
    
  \item [1d.] User id does not exist
  
    - System indicates that the operation failed
\end{enumerate}
\section*{Information Expert}

The information experts for addVersion are System (knows users), User (knows
books) and Book (knows versions).

The addVersion method assigned to System will get the User instance from the
database. The addVersion method assigned to User will then fetch the book
instance from the database. The addVersion method assigned to Book will then add
the version to the database.

\end{document}
